\section{Introduction}

\subsection{Adjoint-based optimization}
Adjoint-based optimization problems typically consist of a simulation
that is run forward in simulation time, producing data that is used in
reverse order by a subsequent adjoint computation that is run
backwards in simulation time.  Figure~\ref{fig:dataflow} shows the
resulting data flow. Besides seismic inversion, many important
numerical problems in science and engineering use adjoints and follow
the same pattern. 

Since the data for each of the computed timesteps in the forward
simulation will be used later in the adjoint computation, it would be
prudent to store it in memory until it is required again, if the required amount
of memory is indeed available. However, the total size of this data
can often run into tens of terabytes and the management of this data
becomes a problem in itself. This data management problem is the
subject of this paper. 

\begin{figure}
\begin{center}
\begin{tikzpicture}
  [node distance=.5cm,
  start chain=1 going right,start chain=2 going left]
     \node[punktchain, join, on chain=1] (t1) {$F(0)$};
     \node[punktchain, join, on chain=1] (t2)      {$F(1)$};
     \node[punktchain, join, on chain=1] (t3)      {$F(2)$};
     \node[punktchain, join, on chain=1] (t4) {$F(3)$};
     \node[punktchain, join, on chain=1] (t5) {$F(4)$};
     \node[punktchain, join, on chain=1] (t6) {$F(5)$};
     \node[punktchain, join, draw=none, on chain=1] (ellipsis1) {$\cdots$};
     \node[punktchain, join, on chain=1] (tn) {$F(n)$};

\node[punktchain, below=1cm of tn, on chain=2] (atn) {$R(n)$};
\node[punktchain, join, draw=none, on chain=2] (ellipsis2) {$\cdots$};
     \node[punktchain, join, on chain=2] (at6)      {$R(5)$};
     \node[punktchain, join, on chain=2] (at5)      {$R(4)$};
     \node[punktchain, join, on chain=2] (at4) {$R(3)$};
     \node[punktchain, join, on chain=2] (at3) {$R(2)$};
     \node[punktchain, join, on chain=2] (at2) {$R(1)$};
     \node[punktchain, join, on chain=2] (at1) {$R(0)$};

\draw[|-,-|,->, thick,] (t1.south) |-+(0,-1em)-| (at1.north);
\draw[|-,-|,->, thick,] (t2.south) |-+(0,-1em)-| (at2.north);
\draw[|-,-|,->, thick,] (t3.south) |-+(0,-1em)-| (at3.north);
\draw[|-,-|,->, thick,] (t4.south) |-+(0,-1em)-| (at4.north);
\draw[|-,-|,->, thick,] (t5.south) |-+(0,-1em)-| (at5.north);
\draw[|-,-|,->, thick,] (t6.south) |-+(0,-1em)-| (at6.north);
\draw[|-,-|,->, thick,] (tn.south) |-+(0,-1em)-| (atn.north);
  \end{tikzpicture}
\end{center}
\caption{The dataflow pattern that is typical of adjoint-based optimization problems}
\label{fig:dataflow}
\end{figure}

\subsection{Example adjoint problem: Seismic inversion}
Seismic inversion typically involves the simulation of the propagation
of seismic waves through the earth's subsurface, followed by a
comparison with data from field measurements. The model of the
subsurface is iteratively improved by minimizing the misfit between
simulated data and field measurement in an adjoint optimization
problem.  Figure~\ref{fig:offshore_survey} shows the setup of the
field experiment that produces the measured
data~\cite{plessix2006review}. 


The data collected in an offshore survey typically consists of a
number of ``shots'' - each of these shots corresponding to different
locations of sources and receivers. As a loose analogy with machine
learning, these correspond to different data points. Since the
gradient computation over a single shot is complex enough that a
single shot can occupy a complete node for $~10^1-10^2$ minutes, the gradient is computed for each of these
shots independently and then collated across all the shots to form a
single update that is used to update the model. It might be evident
here that the processing across shots is easy to parallelise since it
requires a small amount of communication followed by a relatively long
period of independent computation as part of a single iteration of the
optimisation. Since the number of shots is typically of the order of
$10^4$, this offers ample opportunity to fill up a large cluster with
computation, even if an individual shot is only processed on a single
node of the cluster at a time. Hence, it might be worthy to note that
even though this paper focusses on a single gradient evaluation for a
single shot, the overall problem involves carrying out $~10^5$ such
evaluations and is typically run on large clusters for non-trivial
amounts of time.

The first part of seismic inversion, i.e. the simulation of seismic
wave propagation through the earth's subsurface is typically done
using a finite-difference solver and is called the ``forward problem''
in the context of inversion. Looking at this part in isolation, the
data flow here looks like the one shown in
figure~\ref{fig:dataflowfw}. This illustration assumes a first-order
time-stepper, i.e. the computation of each timestep only depends on
the previous timestep. In such a scenario, only two timesteps need to
be kept in memory - the last computed step and the one currently being
computed. In case of an n-th order time-stepper, $(n+1)$ timesteps
need to be kept in memory at any one time. Hence, the memory
requirements of the forward problem can remain constant, regardless of
the number of timesteps the simulation may be run for.

\begin{figure*}
\begin{center}
\begin{tikzpicture}
  [node distance=.5cm,
  start chain=1 going right,start chain=2 going left]
     \node[punktchain, join, on chain=1] (t1) {$F(0)$};
     \node[punktchain, join, on chain=1] (t2)      {$F(1)$};
     \node[punktchain, join, on chain=1] (t3)      {$F(2)$};
     \node[punktchain, join, on chain=1] (t4) {$F(3)$};
     \node[punktchain, join, on chain=1] (t5) {$F(4)$};
     \node[punktchain, join, on chain=1] (t6) {$F(5)$};
     \node[punktchain, join, draw=none, on chain=1] (ellipsis1) {$\cdots$};
     \node[punktchain, join, on chain=1] (tn) {$F(n)$};

  \end{tikzpicture}
\end{center}
\caption{The dataflow pattern typical of a forward-only
  simulation. Boxes represent data and arrows represent computation.}
\label{fig:dataflowfw}
\end{figure*}


TODO remove contractions

\subsection{Memory requirements}
A number of strategies are regularly employed to deal with this
enormous volume of data - the simplest of these being to store it to a
disk, to be read later by the adjoint pass in reverse order. However,
typically the computation to be done on this data in the adjoint phase
takes much less time than the time taken to read it from the disk,
reading from the disk becomes the bottleneck for most practical cases,
rendering this method, although possibly the simplest, also the
slowest of the possible alternatives. Seeing this from the perspective
where thousands of such computations might be running in parallel on a
single cluster (for different shots), the network bandwidth might
restrict the use of a network storage further, hence only node-local disks
may be suitable for this strategy. 

Domain decomposition, where a single shot may be distributed across more
than one node, is often used not only to distribute the computational
workload across more processors, but also to utilize the large amount of memory
available in distributed systems. While this strategy is very powerful, the
number of compute nodes and therefore the amount of memory that can be used
efficiently is limited, for example by communication overheads that start to
dominate as the domain is split into increasingly small
pieces~\cite{virieux2009seismic}. Secondly, this strategy can lead to
a wastage of resources and sometimes a longer time-to-solution for a
given inversion problem using a given number of nodes, especially when
the number of nodes is less than the number of shots. For example, a
problem setup that requires only 10\% more memory than is available on
a single node might not be a good candidate for domain decomposition
over multiple nodes. Lastly, this method is even less applicable on
cloud-based setups since it can be drastically more complicated
to setup and slower due to the communication.


Another common strategy in seismic inversion is to only store values
at the boundaries of the domain at each timestep, and reconstruct the
rest of the wavefield when
required~\cite{clapp2009reverse,yang2014rtm} with time reversal of the
wave equation. However, this method is not applicable for wave
equations that are not time reversible when for example physical
attenuation is included.

Checkpointing is yet another strategy to reduce the memory
overhead. Only a subset of the timesteps during the forward pass is
stored (and the rest discarded). The discarded data is recomputed when
needed by restarting the forward pass from the last available stored
state. We discuss this strategy in section~\ref{revolve}.

Another strategy commonly employed to reduce the memory footprint of
such applications is data compression. This is discussed in
section~\ref{compression}. 

In this paper, we extend the previous studies by \emph{combining} checkpointing
and compression. This is obviously useful when the data does not fit in the
available memory even after compression, for example for very large adjoint
problems, or for problems where the required accuracy limits the achievable
compression ratios.

Compared to the use of only checkpointing without compression, this
combined method often improves performance. This is a consequence of
the reduced size of stored timesteps, allowing more timesteps to be
stored during the forward computation. This in turn reduces the amount
of recomputation that needs to be performed. On the other hand, the
compression and decompression itself takes time. The answer to the
question ``does compression pay off?'', depends on a number of factors including - available
memory, the required precision, the time taken to compress and
decompress, and the achieved compression factors, and various problem specific
parameters like computational intensity of the kernel involved in the
forward and adjoint computations, and the number of timesteps.

Hence, the answer to the compression question depends not only on the
problem one is solving (within seismic inversion, there are numerous
variations of the wave equation that may be solved), but also the
hardware specifics of the machine on which it is being solved. In
fact, as we will see in section TODO, the answer might even change
during the solution process of an individual problem. This brings up
the need to be able to predict whether compression would pay off in a
given scenario, without incurring significant overheads in answering
this question. In this paper, we present the use of a performance
model to answer that question.

\subsection{Summary of contributions}
\begin{itemize}
\item Study the use of different compression algorithms to seismic
  data including 6 lossless and the two most popular lossy compression
  algorithms for floating point data. 
\item Study the compressibility behaviour of seismic data in terms of
  compression factors and compression and decompression times under
  these algorithms
\item Study the performance model for Revolve alone, that takes into
  account time taken to read and write checkpoints
\item An online performance model to predict whether compression would speed up
  an optimization problem
\end{itemize}
