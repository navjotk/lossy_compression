\documentclass[conference]{IEEEtran}
%\IEEEoverridecommandlockouts
% The preceding line is only needed to identify funding in the first footnote. If that is unneeded, please comment it out.
\usepackage{cite}
\usepackage{amsmath,amssymb,amsfonts}
\usepackage{algorithmic}
\usepackage{graphicx}
\usepackage{textcomp}
\usepackage{xcolor}
\def\BibTeX{{\rm B\kern-.05em{\sc i\kern-.025em b}\kern-.08em
    T\kern-.1667em\lower.7ex\hbox{E}\kern-.125emX}}
\begin{document}

\title{Combining checkpointing and data compression\\ for high frequency seismic imaging
}

\author{\IEEEauthorblockN{1\textsuperscript{st} Given Name Surname}
\IEEEauthorblockA{\textit{dept. name of organization (of Aff.)} \\
\textit{name of organization (of Aff.)}\\
City, Country \\
email address}
\and
\IEEEauthorblockN{2\textsuperscript{nd} Given Name Surname}
\IEEEauthorblockA{\textit{dept. name of organization (of Aff.)} \\
\textit{name of organization (of Aff.)}\\
City, Country \\
email address}
\and
\IEEEauthorblockN{3\textsuperscript{rd} Given Name Surname}
\IEEEauthorblockA{\textit{dept. name of organization (of Aff.)} \\
\textit{name of organization (of Aff.)}\\
City, Country \\
email address}
\and
\IEEEauthorblockN{4\textsuperscript{th} Given Name Surname}
\IEEEauthorblockA{\textit{dept. name of organization (of Aff.)} \\
\textit{name of organization (of Aff.)}\\
City, Country \\
email address}
\and
\IEEEauthorblockN{5\textsuperscript{th} Given Name Surname}
\IEEEauthorblockA{\textit{dept. name of organization (of Aff.)} \\
\textit{name of organization (of Aff.)}\\
City, Country \\
email address}
\and
\IEEEauthorblockN{6\textsuperscript{th} Given Name Surname}
\IEEEauthorblockA{\textit{dept. name of organization (of Aff.)} \\
\textit{name of organization (of Aff.)}\\
City, Country \\
email address}
}

\maketitle

\begin{abstract}
Full-waveform inversion is an adjoint-based optimization problem in
seismic imaging that processes terabytes of data, often beyond the
memory capacity of most modern computer systems. A common technique to
carry out the adjoint computation in an environment with
less-than-optimal memory is checkpoint-recomputation. In this
technique, a subset of the forward-computation states are stored in
memory checkpoints and the others are recomputed from the stored
states. The compression of checkpoints can increase the number of
checkpoints available, in a given amount of memory, which, in turn,
can reduce the recompute factor, reducing the time-to-solution of the
adjoint computation. This compression, often comes at an additional
computational cost, and an associated loss of precision that must be
accounted for when choosing between compression schemes. In this
paper, we study the effect of compression on checkpoint-recomputation
methods in memory-constrained scenarios for a realistic FWI workload.
\end{abstract}

\begin{IEEEkeywords}
component, formatting, style, styling, insert
\end{IEEEkeywords}

\section{Introduction}
introduce revolve based checkpointing - cite revolve, pyrevolve

introduction on compression techniques

relate compression to checkpoint compression

Briefly introduce full waveform inversion as an adjoint based optimization
problem. This can be relatively short, as this is only one of many use cases
for our technique.


\section{Related Work}

Towards efficient backward-in-time adjoint computations using data
compression techniques E.C. Cyr, J.N. Shadid, T. Wildey

Lossy Data Compression with DCT Transforms F. Rubio Dalmau,
M. Hanzich, J. de la Puente and N. Gutierrez 

Full seismic waveform tomography for upper-mantle structure in the
Australasian region using adjoint methods
Andreas Fichtner,1 Brian L. N. Kennett,2 Heiner Igel1 and Hans-Peter
Bunge1


\section{Devito, pyRevolve, test case}
short introduction to devito and how to setup a forward-backward
propagator using it - some justification on why this is a realistic
setup even though it might seem like a toy problem

how does pyrevolve implement checkpointing

test case setup.
requirements: should be realistic, numerically stable (to get
realistic compression), big domain,
large number of timesteps
Can we use a SEAM model here?

\section{Compression algorithms}
\subsection{Lossless}
ZLib

variations in compression ratios
\subsection{Lossy}
ZFP
@inproceedings{lossless,
    Acmid = {1126035},
    Address = {Washington, DC, USA},
    Author = {Ratanaworabhan, Paruj and Ke, Jian and Burtscher, Martin},
    Booktitle = {Proceedings of the Data Compression Conference},
    Date-Added = {2015-05-03 20:40:52 +0000},
    Date-Modified = {2018-01-31 10:20:09 +0000},
    Isbn = {0-7695-2545-8},
    Numpages = {10},
    Pages = {133--142},
    Publisher = {IEEE Computer Society},
    Series = {DCC '06},
    Title = {Fast Lossless Compression of Scientific Floating-Point Data},
    Year = {2006}}

@conference{gpu-compression,
    Author = {O'Neil, Molly A. and Burtscher, Martin},
    Booktitle = {Proceedings of the Fourth Workshop on General Purpose Processing on Graphics Processing Units},
    Date-Added = {2015-05-03 22:40:52 +0000},
    Date-Modified = {2015-05-03 22:42:57 +0000},
    Keywords = {GPGPU, floating-point data, lossless data compression, real-time compression},
    Number = {10.1145/1964179.1964189},
    Publisher = {ACM},
    Title = {Floating-point Data Compression at 75 Gb/s on a {GPU}},
    Year = {2011},
}
@inbook{Kaklamanis:2012aa,
    Author = {Kaklamanis, Christos and Papatheodorou, Theodore and Spirakis, PaulG. and Iverson, Jeremy and Kamath, Chandrika and Karypis, George},
    Booktitle = {Euro-Par 2012 Parallel Processing},
    Chapter = {Fast and Effective Lossy Compression Algorithms for Scientific Datasets},
    Da = {2012/01/01},
    Date-Added = {2015-05-03 22:46:47 +0000},
    Date-Modified = {2018-01-31 10:20:09 +0000},
    Isbn = {978-3-642-32819-0},
    La = {English},
    Pages = {843-856},
    Publisher = {Springer Berlin Heidelberg},
    Se = {83},
    Title = {Lecture Notes in Computer Science},
    Title1 = {Fast and Effective Lossy Compression Algorithms for Scientific Datasets},
    Ty = {CHAP},
    Volume = {7484},
    Year = {2012}}

p-adaptive compression from Boehm et al. 
\section{Results}
\subsection{Error analysis}
Discuss the error incurred due to lossy compression. This could be the
angular deviation in the gradient from Boehm et al. as a factor of the
precision parameter, for about 3 different compression schemes  What if a minor
deviation is at a point where you're pointing directly at a
constraint? or a corner point between two constraints? The line search
may mistakenly terminate early. This should be used to derive a bound
for compression ratio based on desired precision. 

\subsection{Compression vs recomputation}

For this paper, we can probably focus on only zfp using 1-3 different precision settings
(nearly lossless, very lossy, something reasonable).

- Time to solution for different compression ratios, for multiple
compression algorithms. at least one lossless compression included
here.

%at least 3 compression algorithms compared

%vary compression ratio within bound established in previous section

\section{Conclusions and Future work}

What are the problems?

Future work:
\begin{itemize}
\item Acceptable error bounds
\item A posteriori and a priori metrics
\item Relationship between compression error and field error, propagation of errors in forward simulation
\item Relationship between compression error and gradient error in cross-correlation
\item What else?
\end{itemize}

Disk checkpointing

\section*{Acknowledgment}

The preferred spelling of the word ``acknowledgment'' in America is without 
an ``e'' after the ``g''. Avoid the stilted expression ``one of us (R. B. 
G.) thanks $\ldots$''. Instead, try ``R. B. G. thanks$\ldots$''. Put sponsor 
acknowledgments in the unnumbered footnote on the first page.

\section*{References}

Please number citations consecutively within brackets \cite{b1}. The 
sentence punctuation follows the bracket \cite{b2}. Refer simply to the reference 
number, as in \cite{b3}---do not use ``Ref. \cite{b3}'' or ``reference \cite{b3}'' except at 
the beginning of a sentence: ``Reference \cite{b3} was the first $\ldots$''

Number footnotes separately in superscripts. Place the actual footnote at 
the bottom of the column in which it was cited. Do not put footnotes in the 
abstract or reference list. Use letters for table footnotes.

Unless there are six authors or more give all authors' names; do not use 
``et al.''. Papers that have not been published, even if they have been 
submitted for publication, should be cited as ``unpublished'' \cite{b4}. Papers 
that have been accepted for publication should be cited as ``in press'' \cite{b5}. 
Capitalize only the first word in a paper title, except for proper nouns and 
element symbols.

For papers published in translation journals, please give the English 
citation first, followed by the original foreign-language citation \cite{b6}.

\begin{thebibliography}{00}
\bibitem{b1} G. Eason, B. Noble, and I. N. Sneddon, ``On certain integrals of Lipschitz-Hankel type involving products of Bessel functions,'' Phil. Trans. Roy. Soc. London, vol. A247, pp. 529--551, April 1955.
\bibitem{b2} J. Clerk Maxwell, A Treatise on Electricity and Magnetism, 3rd ed., vol. 2. Oxford: Clarendon, 1892, pp.68--73.
\bibitem{b3} I. S. Jacobs and C. P. Bean, ``Fine particles, thin films and exchange anisotropy,'' in Magnetism, vol. III, G. T. Rado and H. Suhl, Eds. New York: Academic, 1963, pp. 271--350.
\bibitem{b4} K. Elissa, ``Title of paper if known,'' unpublished.
\bibitem{b5} R. Nicole, ``Title of paper with only first word capitalized,'' J. Name Stand. Abbrev., in press.
\bibitem{b6} Y. Yorozu, M. Hirano, K. Oka, and Y. Tagawa, ``Electron spectroscopy studies on magneto-optical media and plastic substrate interface,'' IEEE Transl. J. Magn. Japan, vol. 2, pp. 740--741, August 1987 [Digests 9th Annual Conf. Magnetics Japan, p. 301, 1982].
\bibitem{b7} M. Young, The Technical Writer's Handbook. Mill Valley, CA: University Science, 1989.
\end{thebibliography}
\vspace{12pt}
\color{red}
IEEE conference templates contain guidance text for composing and formatting conference papers. Please ensure that all template text is removed from your conference paper prior to submission to the conference. Failure to remove the template text from your paper may result in your paper not being published.

\end{document}
