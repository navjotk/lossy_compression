\section{Revolve: Performance model}
\label{sec:revolve}
Checkpointing is a commonly used strategy to reduce the memory footprint
of adjoint problems. Here, depending on the memory available, some timesteps
computed in the forward pass are stored, while others are discarded. The ones
that were discarded are later recomputed by rerunning the forward pass from
the last stored checkpoint. 
The Revolve algorithm~\cite{griewank2000algorithm} provides an
answer to the question of which timesteps should be stored and which
states should be recomputed to minimize the total amount of
recomputation work. Other authors have subsequently developed
extensions to Revolve that are optimal under different
assumptions~\cite{stumm2009multistage,
wang2009minimal,aupy2016optimal,schanen2016asynchronous, aupy2017periodicity}. Previous
work has applied checkpointing to seismic imaging and inversion
problems~\cite{symes2007reverse, datta2018asynchronous}.


In this section, we build on the ideas introduced in \cite{stumm2009multistage} to build a performance model that can be used to predict the runtime of an adjoint computation that uses the Revolve checkpointing strategy. 
We call the time taken by a single forward computational step $C_F$
and correspondingly, the time taken by a single backward step $C_R$. For a simulation with $\mathbf{N}$ timesteps, the minimum wall time required
for the full forward-adjoint evaluation is given by
\begin{equation}
T_N = \mathbf{C_F} \cdot \mathbf{N} + \mathbf{C_R} \cdot \mathbf{N}
\end{equation}
If the size of a single timestep in memory is given by $\mathbf{S}$, this
requires a memory of at least size $\mathbf{S} \cdot \mathbf{N}$. If sufficient memory
is available, no checkpointing or compression is needed.

If the memory is smaller than $\mathbf{S} \cdot \mathbf{N}$, Revolve provides
a strategy to solve for the adjoint field by storing a subset of the $\mathbf{N}$ total checkpoints
and recompute the remaining ones. The overhead introduced by this method can be broken down into
the recomputation overhead $\mathbf{O}_R$ and the storage overhead $\mathbf{O}_S$. The recomputation
overhead is the amount of time spent in recomputation, given by
\begin{equation}
\mathbf{O}_R(N, M) = p(N, M) \cdot \mathbf{C_F},
\end{equation}
where $p(N, M)$ is the minimum number of recomputed steps from \cite{griewank2000algorithm}, reproduced
here in equation \ref{eqn:recompute}.
\begin{figure*}
\begin{equation}
p(N, M) = \begin{cases}
      N(N-1) /2, & \text{if}\ M=1 \\
      \min\limits_{1<=\widetilde{N}<=N} \{\widetilde{N} + p(\widetilde{N}, M) + p(N-\widetilde{N}, M-1)\}, & \text{if}\ M>1
    \end{cases}
    \label{eqn:recompute}
\end{equation}
\end{figure*}
In equation \ref{eqn:recompute}, M is the number of checkpoints that can be
stored in memory. Note that for $M >=N$, $\mathbf{O}_R$ would be zero. For $M <
N$, $\mathbf{O}_R$ grows rapidly as M is reduced relative to N. 

In an ideal implementation, the storage overhead $\mathbf{O}_S$ might be zero, since the computation could
be done ``in-place'', but in practice, checkpoints are generally stored in a separate section of memory and they
need to be transferred to a ``computational'' section of the memory where the computation is performed, and then
the results copied back to the checkpointing memory. This copying is a common feature of checkpointing
implementations, and might pose a non-trivial overhead when the
computation involved in a single timestep is not very large. 
This storage overhead is given by:
\begin{equation}
\mathbf{O}_{SR}(N, M) = \mathbf{W}(N, M) \cdot \frac{\mathbf{S}}{\mathbf{B}} +
\mathbf{N} \cdot \frac{\mathbf{S}}{\mathbf{B}}
\label{eqn:storage}
\end{equation}
where $\mathbf{W}$ is the total number of times Revolve writes
checkpoints for a single run, $ \mathbf{N}$ is the number of times
checkpoints are read, and $\mathbf{B}$ is the bandwidth at which these
copies happen. The total time to solution becomes
\begin{equation}
T_R = \mathbf{C_F} \cdot \mathbf{N} + \mathbf{C_R} \cdot \mathbf{N} + \mathbf{O}_R(N, M) +
\mathbf{O}_{SR}(N, M)
\end{equation} 